Биномиальное распределение - это модель вероятности, которая описывает случайные эксперименты с двумя возможными исходами (обычно "успех" и "неудача"), где вероятность успеха остается постоянной для каждого испытания.

Вот ключевые точки:

1. **Два исхода**: Биномиальное распределение применяется в ситуациях, где событие может иметь только два возможных исхода. Например, монета может выпасть орлом или решкой, испытание может быть успешным или неудачным и т.д.

2. **Фиксированное число испытаний**: Эксперимент повторяется фиксированное количество раз, обозначаемое как \( n \).

3. **Независимые испытания**: Каждое испытание независимо от предыдущих. Результат одного испытания не влияет на результат следующего.

4. **Константная вероятность успеха**: Вероятность успеха для каждого испытания остается постоянной. Обозначается как \( p \).

5. **Вероятность "успеха" и "неудачи"**: Вероятность успеха обычно обозначается как \( p \), а вероятность неудачи (неудача) - как \( q = 1 - p \).

6. **Формула вероятности**: Вероятность \( P(X = k) \) того, что произойдет \( k \) успехов в \( n \) испытаниях, задается формулой биномиального распределения:
\[ P(X = k) = \binom{n}{k} \times p^k \times (1 - p)^{n - k} \]
где \( \binom{n}{k} \) - это число сочетаний \( n \) по \( k \) (или "n по k").

7. **Свойства**: Биномиальное распределение симметрично при \( p = 0.5 \) (вероятности успеха и неудачи равны). С увеличением числа испытаний \( n \) оно становится все более похожим на нормальное распределение.

Итак, биномиальное распределение - это модель, которая помогает предсказать вероятность количества успехов в серии независимых испытаний с двумя исходами.